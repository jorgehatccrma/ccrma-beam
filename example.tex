%!TEX program = lualatex
\documentclass{beamer}

\usetheme[customtitle, gradient, simplefootline]{CCRMA}
% the CCRMA theme offers the following options:
%   dark:         use a dark theme
%   customtitle:  use a custom title page (including author's email if specified)
%   gradient:     use a gradient background
%   simple:       prevent the use of the footer and colored headline
%   simplefootline: show only page numbers in the footline


% AWESOME Beamer Cheat-sheet:
% http://www.cpt.univ-mrs.fr/~masson/latex/Beamer-appearance-cheat-sheet.pdf


% IF USING LUALATEX (OR XELATEX, UNTESTED), IT IS POSSIBLE TO USE SYSTEM FONTS
% THESE ARE SOME EXAMPLES

% \usepackage[bitstream-charter]{mathdesign}
\usepackage{fontspec}

% \setmainfont{TeX Gyre Bonum}
% \setmainfont{TeX Gyre Termes}
\setmainfont{XITS}


% \setsansfont{Source Sans Pro}  % for this you'll need to install "Source Sans Pro" OTF font in your system
% \setsansfont{Open Sans}  % for this you'll need to install "Open Sans" OTF font in your system
% \setsansfont{Raleway}  % for this you'll need to install "Raleway" TTF font in your system
% \setsansfont{Ubuntu}  % for this you'll need to install "Ubuntu" TTF font in your system
% \setsansfont[BoldFont={Ubuntu Bold}, ItalicFont={Ubuntu Light Italic}, BoldItalicFont={Ubuntu Bold Italic}]{Ubuntu Light}  % for this you'll need to install "Ubuntu" TTF font in your system
% \setsansfont{PT Sans}
% \setsansfont{Lato}
% \setsansfont[BoldFont={Colaborate-Bold}, ItalicFont={Colaborate-Thin}, BoldItalicFont={Colaborate-Medium}]{Colaborate-Regular}
% \setsansfont{Aller}
\setsansfont{Asap}

% \setmonofont{Source Code Pro}  % for this you'll need to install "Source Code Pro" OTF font in your system
\setmonofont{Source Code Pro Light}  % for this you'll need to install "Source Code Pro" OTF font in your system
% \setmonofont{Inconsolata}  % for this you'll need to install "Inconsolata" OTF font in your system


\newfontfamily \ccrmaTitleFont {Ubuntu}
% \newfontfamily \ccrmaTitleFont {Amaranth}
% \newfontfamily \ccrmaTitleFont {Asap}
% \newfontfamily \alternativeFont {Raleway}
% \newfontfamily \alternativeFont {Amaranth}
\newfontfamily \alternativeFont {Asap}

\newfontfamily \ccrmaLiteFont {Ubuntu-light}


\setbeamerfont{title}{family=\ccrmaTitleFont}
\setbeamerfont{frametitle}{family=\ccrmaTitleFont}
\setbeamerfont*{enumerate item}{family=\ccrmaTitleFont}
\setbeamerfont*{enumerate subitem}{family=\ccrmaTitleFont}
\setbeamerfont*{enumerate subsubitem}{family=\ccrmaTitleFont}
\setbeamerfont{section in toc}{family=\ccrmaTitleFont}
\setbeamerfont{block title}{family=\alternativeFont}
\setbeamerfont{block title alert}{family=\alternativeFont}
\setbeamerfont{block title example}{family=\alternativeFont}


% \usefonttheme[onlylarge]{structurebold}

\setbeamerfont{footline}{family=\ttfamily}



\renewcommand{\theequation}{{\small \ccrmaLiteFont\arabic{equation}}}
% \renewcommand{\theequation}{{\small \ccrmaTitleFont\arabic{equation}}}
% \renewcommand{\theequation}{{\ttfamily\arabic{equation}}}

\everymath{\color{normal text.fg!70!structure.bg}}


\usepackage{multimedia}
% this line is to fix a weird rendering issue in Adobe Acrobat on frames using the
% \movie tag with a "poster" image with transparency
\pdfpageattr {/Group << /S /Transparency /I true /CS /DeviceRGB>>}


% we'll use this package to allow absolute position (e.g. for images)
% this package can take you a long way for custom layouts, although it will put
% more responsibility on your shoulders. If you want to explore, go to the texpos'
% documentation: http://www.tex.ac.uk/ctan/macros/latex/contrib/textpos/textpos.pdf
\usepackage[absolute,overlay]{textpos}


\usepackage{rotating}


\usepackage{appendixnumberbeamer}





% presentation "meta-data"
\title{CCRMA Beamer Template}
\subtitle{Sample Presentation}
\author{Jorge Herrera}
\authoremail{jorgeh@ccrma.stanford.edu} % specific of the CCRMA beamer theme
\date

\titlepagecontent{\includegraphics[width=2.5cm]{stanford-logo.pdf}}







% Actual presentation
\begin{document}

\begin{frame}[plain]
    \titlepage
\end{frame}
\addtocounter{framenumber}{-1}  % so this page is not considered by the page counter


\begin{frame}[plain]
    \tableofcontents
\end{frame}
\addtocounter{framenumber}{-1}  % so this page is not considered by the page counter

\section[Intro]{Intro}
\begin{frame}[plain]
    \tableofcontents[currentsection]
\end{frame}
\addtocounter{framenumber}{-1}  % so this page is not considered by the page counter

\subsection[basics]{The Basics}

\begin{frame}\frametitle{First page}
    \begin{itemize}
    \item You say \textbf{yes}
        \begin{itemize}
        \item ...
        \end{itemize}
    \item I say \\
    \centering
    $e^{j \pi} == 1?$
    \end{itemize}

\end{frame}

\begin{frame}\frametitle{Second page}
    \begin{enumerate}
    \item You say goodbye
    \item and I say

    \begin{equation}
    X(\omega_k) = \sum_{n=0}^{N-1}x(n)e^{-j2\pi kn/N}
    \end{equation}

    \begin{equation}
    z \in \mathbb{C}
    \end{equation}

    \end{enumerate}
\end{frame}


\subsection[av]{Including Audio/Video}

\begin{frame}\frametitle{Further details}
{
\centering
Multimedia package test: can you hear this?

$\downarrow$

\movie[showcontrols=true]{\includegraphics[height=6ex]{sound.pdf}}{154581__ecfike__i-m-all-right.wav}
% \movie[showcontrols=true]{\includegraphics[height=3ex]{movie.pdf}\hspace{4em}}{154581__ecfike__i-m-all-right.wav}
% \sound[beatspersample=16, channels=2, samplingrate=44100, encoding=Raw]{\includegraphics[height=8ex]{sound.pdf}}{154581__ecfike__i-m-all-right.wav}

\scriptsize
Since so far only Adobe has implemented embedded AV playback, you will need to use Acrobat as the viewer (haven't tested on Windows, but on OS X neither Preview nor Skim could play them).

}


\end{frame}


\section[Outro]{Outro}
\begin{frame}[plain]
    \tableofcontents[currentsection]
\end{frame}
\addtocounter{framenumber}{-1}  % so this page is not considered by the page counter

\begin{frame}

    \begin{block}{General block title}
    An idea\footnote{don't use titles if not needed}
    \end{block}

    \begin{example}[ex. 1]
    For example ...
    \end{example}

    \begin{alertblock}{Alert title}
    \centering
    \textbf{Or an alert!}
    \end{alertblock}

\end{frame}


\begin{frame}[plain]{Example of a table}

    \vspace{2cm}

    \begin{center}
        \begin{tabular}{c c c}
        \bf Name & \bf x & \bf y \\
        \hline
        $\alpha$ & 8 & 10 \\
        $\eta$ & 9 & 9 \\
        \hline
        \end{tabular}
    \end{center}

    \vspace{2cm}

    \begin{center}
    \scriptsize
    \hfill You can also omit the footer of a page
    \end{center}

\end{frame}

\begin{frame}{How to include an image}

    You can insert images the traditional \LaTeX way

    \begin{figure}
        \includegraphics[width=0.3\textwidth]{stanford-logo.pdf}
        \caption{Who's logo is this?}
    \end{figure}
\end{frame}


\begin{frame}{Absolute positioning}

    Or you can use absolute positioning, via the \texttt{texpos} package

  % These vars may be altered to affect how texpos package (e.g. \begin{textblock} \end{textblock}) is used
  % \setlength{\TPHorizModule}{1cm}
  % \setlength{\TPVertModule}{1cm}


    \begin{textblock}{20}(6,11)
    \begin{rotate}{33}
      \includegraphics[width=0.3\textwidth]{ccrma-logo.pdf}
    \end{rotate}
    \end{textblock}


\end{frame}


\begin{frame}{Using columns in a frame}
    \begin{columns}[t]
        \column{0.4\textwidth}
            This is content of the first column. You can use as many columns as you like, using the \texttt{columns} environment in \texttt{beamer}
        \column{0.4\textwidth}
            In the second column you can do whatever you want, like adding an image for example.
            \includegraphics[width=\textwidth]{ccrma-logo.pdf}
    \end{columns}
\end{frame}



\appendix  % this generates a warning, but it doesn't seem to affect the document visually

\begin{frame}{First Extra slide}
This is an extra slide, so it won't affect the page count displayed at the bottom of the main presentation. Basically, the page counter is reset here.
\end{frame}


\end{document}