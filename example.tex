\documentclass{beamer}

\usetheme[]{CCRMA}    % you can specify the option 'dark' to use a dark background

% presentation "meta-data"
\title{A sample presentation}
\subtitle{Using the CCRMA theme, with a very long subtitle to test how it gets displayed}
\author{Jorge Herrera}
\authoremail{jorgeh@ccrma.stanford.edu} % specific of the CCRMA beamer theme
\date



% Actual presentation
\begin{document}

\begin{frame}[plain]
\titlepage
\end{frame}
\addtocounter{framenumber}{-1}  % so this page is not considered by the page counter


\begin{frame}[plain]
    \tableofcontents
\end{frame}

\section[Intro]{Introduction}
\begin{frame}[plain]
    \tableofcontents[currentsection]
\end{frame}

\begin{frame}\frametitle{First page}
    \begin{itemize}
    \item You say yes
        \begin{itemize}
        \item sometimes
            \begin{itemize}
            \item Testing a third level
            \item (I don't feel very creative right now)
            \end{itemize}
        \end{itemize}
    \item I say no
    \end{itemize}

    \begin{equation}
    \frac{\partial \theta}{\partial x} = \lambda \sin(\alpha x)
    \end{equation}

\end{frame}

\begin{frame}\frametitle{Second page}
    \begin{enumerate}
    \item You say goodbye
    \item and I say hello
        \begin{enumerate}
        \item Second level enumeration
            \begin{enumerate}
            \item Third level
            \end{enumerate}
        \end{enumerate}
    \end{enumerate}
\end{frame}

\begin{frame}\frametitle{Further details}
Testing
\end{frame}


\section[Cont]{Continuation}
\begin{frame}[plain]
    \tableofcontents[currentsection]
\end{frame}

\subsection[First]{First Part}
\begin{frame}\frametitle{Another page}

    \begin{block}{Block title}
    Block content
    \end{block}

    \begin{example}{Example title}
    Example content
    \end{example}

    \begin{alertblock}{Alert title}
    Alert content
    \end{alertblock}

\end{frame}


\subsection[Second]{Second Part}
\begin{frame}{Example of a table}
    \begin{center}
        \begin{tabular}{l c c}
        \bf Name & \bf RMS & \bf Total \\
        \hline
        Nando & 8 & 10 \\
        Carr & 9 & 9 \\
        \hline
        \end{tabular}
    \end{center}
\end{frame}

\begin{frame}{How to include an image}
    \begin{figure}
        \includegraphics[width=0.5\textwidth]{stanford-logo.pdf}
        \caption{Who's logo is this?}
    \end{figure}
\end{frame}

\begin{frame}{Using columns in a frame}
    \begin{columns}[t]
        \column{0.4\textwidth}
            This is content of the first column. You can use as many columns as you like, using the \texttt{columns} environment in \texttt{beamer}
        \column{0.4\textwidth}
            In the second column you can do what you like, like adding an image, for instance.
            \includegraphics[width=\textwidth]{ccrma-logo.pdf}
    \end{columns}
\end{frame}

\end{document}