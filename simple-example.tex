%!TEX program = lualatex
\documentclass{beamer}

\usetheme[customtitle,
         simplefootline,
         simple,
         dark,
         centerframetitle,
         corners=1pt,
         blockpadding=0.4em,
         shadowedblocks,
         outlineblocks
         ]{CCRMA}
% the CCRMA theme offers the following options:
%   dark:         use a dark theme
%   customtitle:  use a custom title page (including author's email if specified)
%   gradient:     use a gradient background
%   simple:       prevent the use of the footer and colored headline
%   simplefootline: show only page numbers in the footline
%   centerframetitle: center frame titles
%   centerauthorinfo: center author (and related info) in the title page
%   corners:      block corner radius (e.g. corners=5pt)
%   blockpadding: block top and bottom padding (e.g. blockpadding=0.2em)
%   shadowedblocks: display shadowed blocks
%   outlineblocks: draw outlines for blocks


% AWESOME Beamer Cheat-sheet:
% http://www.cpt.univ-mrs.fr/~masson/latex/Beamer-appearance-cheat-sheet.pdf






\usepackage{multimedia}
% this line is to fix a weird rendering issue in Adobe Acrobat on frames using the
% \movie tag with a "poster" image with transparency
\pdfpageattr {/Group << /S /Transparency /I true /CS /DeviceRGB>>}


% we'll use this package to allow absolute position (e.g. for images)
% this package can take you a long way for custom layouts, although it will put
% more responsibility on your shoulders. If you want to explore, go to the texpos'
% documentation: http://www.tex.ac.uk/ctan/macros/latex/contrib/textpos/textpos.pdf
\usepackage[absolute,overlay]{textpos}









%%%%%%%%%%%%%%%%%%%%%%%%%%%%%%%%%%%%%%%%%%%%%%%%%%%%%%%%%%%%%%%%%%%%%%%%%%%%%%%%
%%%%%%%%%%%%%%%%%%%%%%%%%%%%%%%%%%%%%%%%%%%%%%%%%%%%%%%%%%%%%%%%%%%%%%%%%%%%%%%%
% CONTENT STARTS HERE
%%%%%%%%%%%%%%%%%%%%%%%%%%%%%%%%%%%%%%%%%%%%%%%%%%%%%%%%%%%%%%%%%%%%%%%%%%%%%%%%
%%%%%%%%%%%%%%%%%%%%%%%%%%%%%%%%%%%%%%%%%%%%%%%%%%%%%%%%%%%%%%%%%%%%%%%%%%%%%%%%


%%%%%%%%%%%%%%%%%%%%%%%%%%%%%%%%%%%%%%%%%%%%%%%%%%%%%%%%%%%%%%%%%%%%%%%%%%%%%%%%
% Title page info
%%%%%%%%%%%%%%%%%%%%%%%%%%%%%%%%%%%%%%%%%%%%%%%%%%%%%%%%%%%%%%%%%%%%%%%%%%%%%%%%

% presentation "meta-data"
\title{CCRMA Beamer Template}
\subtitle{Simple Presentation}
\author{Author One \\ Author Two \\ Author Three}
% \authoremail{author1@ccrma.stanford.edu} % specific of the CCRMA beamer theme
\date

\institute[CCRMA]{
    % Center for Computer Research in Music and Acoustics\\[0.5ex]
    CCRMA\\
    Department of Music\\
    Stanford University%\\
    %Stanford CA, 94305
}

\date{\today}

\titlepageinstituelogo{
    \includegraphics[width=2.5cm]{stanford-ccrma-logo.pdf}
}


\titlegraphic{
    \includegraphics[width=2.5cm]{stanford-logo.pdf}
}




%%%%%%%%%%%%%%%%%%%%%%%%%%%%%%%%%%%%%%%%%%%%%%%%%%%%%%%%%%%%%%%%%%%%%%%%%%%%%%%%
% SLIDES
%%%%%%%%%%%%%%%%%%%%%%%%%%%%%%%%%%%%%%%%%%%%%%%%%%%%%%%%%%%%%%%%%%%%%%%%%%%%%%%%

% Actual presentation
\begin{document}

\begin{frame}[plain]
    \titlepage
\end{frame}
\addtocounter{framenumber}{-1}  % so this page is not considered by the page counter


\begin{frame}[plain]
    \tableofcontents
\end{frame}
\addtocounter{framenumber}{-1}  % so this page is not considered by the page counter

\section[Intro]{Intro}

\begin{frame}[plain]
    \tableofcontents[currentsection]
\end{frame}
\addtocounter{framenumber}{-1}  % so this page is not considered by the page counter



\subsection[av]{Including Audio/Video}

\begin{frame}\frametitle{Further details}\framesubtitle{(about video)}
{
\centering
Multimedia package test: can you hear this?

$\downarrow$

\movie[showcontrols=true]{\includegraphics[height=6ex]{sound.pdf}}{154581__ecfike__i-m-all-right.wav}
% \movie[showcontrols=true]{\includegraphics[height=3ex]{movie.pdf}\hspace{4em}}{154581__ecfike__i-m-all-right.wav}
% \sound[beatspersample=16, channels=2, samplingrate=44100, encoding=Raw]{\includegraphics[height=8ex]{sound.pdf}}{154581__ecfike__i-m-all-right.wav}

\scriptsize
Since so far only Adobe has implemented embedded AV playback, you will need to use Acrobat as the viewer (haven't tested on Windows, but on OS X neither Preview nor Skim could play them).
}

\end{frame}


\section[Outro]{Outro}

\begin{frame}[plain]
    \tableofcontents[currentsection]
\end{frame}
\addtocounter{framenumber}{-1}  % so this page is not considered by the page counter



\begin{frame}{Absolute positioning}

    Or you can use absolute positioning, via the \texttt{texpos} package

  % These vars may be altered to affect how texpos package (e.g. \begin{textblock} \end{textblock}) is used
  % \setlength{\TPHorizModule}{1cm}
  % \setlength{\TPVertModule}{1cm}


    \begin{textblock}{20}(6,11)
      \includegraphics[width=0.3\textwidth]{ccrma-logo.pdf}
    \end{textblock}


\end{frame}



\begin{frame}{Using columns in a frame}
    \begin{columns}[t]
        \column{0.4\textwidth}
            This is content of the first column. You can use as many columns as you like, using the \texttt{columns} environment in \texttt{beamer}
        \column{0.4\textwidth}
            In the second column you can do whatever you want, like adding an image for example.
            \includegraphics[width=\textwidth]{ccrma-logo.pdf}
    \end{columns}
\end{frame}



\begin{frame}
    Showing bullets one at a time
    \begin{itemize}[<+->]
        \item Item 1
        \item Item 2
    \end{itemize}
\end{frame}




\begin{frame}
    Showing bullets one at a time, with a custom text for each
    \begin{itemize}[<+->]
        \item Item 1
        \item Item 2
    \end{itemize}

    \begin{block}{Definition}
        \only<1>{Custom text for item 1.}
        \only<2>{Custom text for item 2.}
    \end{block}

\end{frame}



\end{document}